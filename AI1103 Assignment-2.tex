\documentclass[journal,12pt,twocolumn]{IEEEtran}

\usepackage{setspace}
\usepackage{gensymb}
\singlespacing
\usepackage[cmex10]{amsmath}

\usepackage{amsthm}

\usepackage{mathrsfs}
\usepackage{txfonts}
\usepackage{stfloats}
\usepackage{bm}
\usepackage{cite}
\usepackage{cases}
\usepackage{subfig}

\usepackage{longtable}
\usepackage{multirow}

\usepackage{enumitem}
\usepackage{mathtools}
\usepackage{steinmetz}
\usepackage{tikz}
\usepackage{circuitikz}
\usepackage{verbatim}
\usepackage{tfrupee}
\usepackage[breaklinks=true]{hyperref}
\usepackage{graphicx}
\usepackage{tkz-euclide}

\usetikzlibrary{calc,math}
\usepackage{listings}
\usepackage{color}                                            %%
\usepackage{array}                                            %%
\usepackage{longtable}                                        %%
\usepackage{calc}                                             %%
\usepackage{multirow}                                         %%
\usepackage{hhline}                                           %%
\usepackage{ifthen}                                           %%
\usepackage{lscape}     
\usepackage{multicol}
\usepackage{chngcntr}

\DeclareMathOperator*{\Res}{Res}

\renewcommand\thesection{\arabic{section}}
\renewcommand\thesubsection{\thesection.\arabic{subsection}}
\renewcommand\thesubsubsection{\thesubsection.\arabic{subsubsection}}

\renewcommand\thesectiondis{\arabic{section}}
\renewcommand\thesubsectiondis{\thesectiondis.\arabic{subsection}}
\renewcommand\thesubsubsectiondis{\thesubsectiondis.\arabic{subsubsection}}


\hyphenation{op-tical net-works semi-conduc-tor}
\def\inputGnumericTable{}                                 %%

\lstset{
	%language=C,
	frame=single, 
	breaklines=true,
	columns=fullflexible
}
\begin{document}
	
	
	\newtheorem{theorem}{Theorem}[section]
	\newtheorem{problem}{Problem}
	\newtheorem{proposition}{Proposition}[section]
	\newtheorem{lemma}{Lemma}[section]
	\newtheorem{corollary}[theorem]{Corollary}
	\newtheorem{example}{Example}[section]
	\newtheorem{definition}[problem]{Definition}
	
	\newcommand{\BEQA}{\begin{eqnarray}}
		\newcommand{\EEQA}{\end{eqnarray}}
	\newcommand{\define}{\stackrel{\triangle}{=}}
	\bibliographystyle{IEEEtran}
	\raggedbottom
	\setlength{\parindent}{0pt}
	\providecommand{\mbf}{\mathbf}
	\providecommand{\pr}[1]{\ensuremath{\Pr\left(#1\right)}}
	\providecommand{\qfunc}[1]{\ensuremath{Q\left(#1\right)}}
	\providecommand{\sbrak}[1]{\ensuremath{{}\left[#1\right]}}
	\providecommand{\lsbrak}[1]{\ensuremath{{}\left[#1\right.}}
	\providecommand{\rsbrak}[1]{\ensuremath{{}\left.#1\right]}}
	\providecommand{\brak}[1]{\ensuremath{\left(#1\right)}}
	\providecommand{\lbrak}[1]{\ensuremath{\left(#1\right.}}
	\providecommand{\rbrak}[1]{\ensuremath{\left.#1\right)}}
	\providecommand{\cbrak}[1]{\ensuremath{\left\{#1\right\}}}
	\providecommand{\lcbrak}[1]{\ensuremath{\left\{#1\right.}}
	\providecommand{\rcbrak}[1]{\ensuremath{\left.#1\right\}}}
	\theoremstyle{remark}
	\newtheorem{rem}{Remark}
	\newcommand{\sgn}{\mathop{\mathrm{sgn}}}
	\providecommand{\abs}[1]{\(\left\vert#1\right\vert\)}
	\providecommand{\res}[1]{\Res\displaylimits_{#1}} 
	\providecommand{\norm}[1]{\(\left\lVert#1\right\rVert\)}
	%\providecommand{\norm}[1]{\lVert#1\rVert}
	\providecommand{\mtx}[1]{\mathbf{#1}}
	\providecommand{\mean}[1]{E\(\left[ #1 \right]\)}
	\providecommand{\fourier}{\overset{\mathcal{F}}{ \rightleftharpoons}}
	%\providecommand{\hilbert}{\overset{\mathcal{H}}{ \rightleftharpoons}}
	\providecommand{\system}{\overset{\mathcal{H}}{ \longleftrightarrow}}
	%\newcommand{\solution}[2]{\textbf{Solution:}{#1}}
	\newcommand{\solution}{\noindent \textbf{Solution: }}
	\newcommand{\cosec}{\,\text{cosec}\,}
	\providecommand{\dec}[2]{\ensuremath{\overset{#1}{\underset{#2}{\gtrless}}}}
	\newcommand{\myvec}[1]{\ensuremath{\begin{pmatrix}#1\end{pmatrix}}}
	\newcommand{\mydet}[1]{\ensuremath{}}
	\numberwithin{equation}{subsection}
	\makeatletter
	\@addtoreset{figure}{problem}
	\makeatother
	\let\StandardTheFigure\thefigure
	\let\vec\mathbf
	\renewcommand{\thefigure}{\theproblem}
	\def\putbox#1#2#3{\makebox[0in][l]{\makebox[#1][l]{}\raisebox{\baselineskip}[0in][0in]{\raisebox{#2}[0in][0in]{#3}}}}
	\def\rightbox#1{\makebox[0in][r]{#1}}
	\def\centbox#1{\makebox[0in]{#1}}
	\def\topbox#1{\raisebox{-\baselineskip}[0in][0in]{#1}}
	\def\midbox#1{\raisebox{-0.5\baselineskip}[0in][0in]{#1}}
	\vspace{3cm}
	\title{AI1103 - Assignment 2}
	\author{I.Rajasekhar Reddy -- CS20BTECH11020}
	\maketitle
	\newpage
	\bigskip
	\renewcommand{\thefigure}{\theenumi}
	\renewcommand{\thetable}{\theenumi}
	Download all python codes from 
	\begin{lstlisting}
	........................................\\ \\
	\end{lstlisting}
	%
	and latex-tikz codes from 
	%
	\begin{lstlisting}
	.........................................\\ \\
	\end{lstlisting}
	QUESTION:\\
	If P and Q are two random events, then the
	following is TRUE:\\ \\
	(A) Independence of P and Q implies that $\pr{P \cap Q}$ $=$ 0 \\ \\
	(B) $\pr{P \cup Q}$ $\geq$ $\pr{P} + \pr{Q}$ \\ \\
	(C) If P and Q are mutually exclusive, then they must be independent.\\ \\
	(D) $\pr{P \cap Q} \leq \pr{P}$\\ \\
	ANSWER:\\
	(A) Independence of P and Q means if P happens, then outcome of Q won't be affected by that.\\
	so \begin{align}
		\pr{P/Q} &= \pr{P} \\
		\frac{\pr{P\cap Q}}{\pr{Q}} &= \pr{P} \\
		\implies  \pr{P \cap Q} &= \pr{P}.\pr{Q}
	\end{align}
	This is what we can say hence (A) is wrong \\ \\
	(B)As \begin{align}
		\pr{P \cup Q} &= \pr{P} + \pr{Q} -\pr{P \cap Q} \\
		\pr{P \cup Q} + \pr{P \cap Q} &= \pr{P} + \pr{Q} \\
		\pr{P \cap Q} &\geq 0 \\
		\implies \pr{P} + \pr{Q} &\geq \pr{P \cup Q}
	\end{align}
	Hence (B) is also wrong \\ \\  \\
	(C) When P and Q are mutually exclusive, then either P occurs or Q occurs but not both simultaneously. So if P happens, chance of Q happening gets ruled out and vice-versa. Hence, mutually exclusive events are dependent.\\Hence (C) is also wrong\\ \\
	(D)As \begin{align}
		\pr{Q/P} &= \frac{\pr{P \cap Q}}{\pr{P}} 
	\end{align}
	And \begin{align}
		\pr{Q/P} &\leq 1\\
		\frac{\pr{P \cap Q}}{\pr{P}} &\leq 1\\ 
		\pr{P \cap Q} &\leq \pr{P}
	\end{align}
	Hence (D) is correct.
	
	
	
	
	
\end{document}


